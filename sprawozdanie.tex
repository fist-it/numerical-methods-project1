\documentclass[a4paper,12pt]{article}  % or "report" for larger documents

\usepackage{polski}  % Polish language support
\usepackage[utf8]{inputenc}  % UTF-8 encoding
\usepackage{amsmath, amssymb}  % Math support
\usepackage{float}
\usepackage{graphicx}  % Insert images
\usepackage{hyperref}  % Clickable links
\usepackage{biblatex}  % Bibliography management
\addbibresource{references.bib}  % Reference file
\renewcommand{\figurename}{rys.}

\title{Metody numeryczne, projekt1:\\ wskaźnik MACD}
\author{Franciszek Fabinski - s197797}
\date{\today}

\begin{document}

\maketitle  % Generates title page

\section{Wstep teoretyczny}
Wskaźnik MACD (Moving Average Convergence Divergence) jest jednym z
najpopularniejszych wskaźników analizy technicznej. Stworzony przez 
Geralda Appela w latach 70-tych XX wieku, miał pomagać inwestorom
w analizie trendów na rynkach finansowych. 

Wskaźnik MACD wynika z porównania dwóch wykładniczych średnich kroczących (EMA), przeważnie
jednej krótszej (szybciej reagującej na zmiany cen) i jednej dłuższej
(reagującej wolniej). Pozwala to na wykrycie zmian trendu na rynku.

Poza samą linią MACD, często stosuje się też linie sygnałową powstajacą
z wyznaczenia wykładniczej średniej kroczącej z linii MACD. Przecięcie tych linii 
sugeruje sygnał kupna lub sprzedaży.

\section{Formalizm matematyczny i dane testowe}
Zaimplementowany wskaźnik MACD z definicji korzysta z dwóch średnich kroczących:
\begin{equation}
    MACD = EMA_{12} - EMA_{26}
\end{equation}
gdzie $EMA_{n}$ to średnia krocząca z $n$ okresów.
W przypadku mojej pracy jeden okres jest równy jednemu dniu.
Linia SIGNAL jest za to określona jako
\begin{equation}
    SIGNAL = EMA_{9}(MACD)
\end{equation}


\pagebreak


Do obliczenia wykładniczej średniej kroczącej z $n$ okresów korzystam z niżej
podanego wzoru:
\begin{equation}
    EMA_{n}(i) = \alpha \cdot x_i + (1 - \alpha) \cdot EMA_{n}(i-1)
\end{equation}


Dane równanie jestesmy w stanie przekształcić do postaci jawnej:

\begin{equation}
  EMA_n(i) = \frac{x_i + (1-\alpha)x_{i-1} + (1-\alpha)^2 x_{i-2} + ... +
  (1-\alpha)^i x_0}{1 + (1-\alpha) + (1-\alpha)^2 + ... + (1-\alpha)^i}
\end{equation}

gdzie:
\begin{itemize}
  \item $EMA_{n}(i)$ to wartosc średniej kroczącej z $n$ okresów w $i$-tym okresie
  \item $x_i$ to wartosc z danego okresu
  \item $\alpha = \frac{2}{n+1}$
\end{itemize}


Postać jawna równania (4) sugeruje, ze jesteśmy w stanie obliczyć 
$EMA_{n}(0)$ juz z jednej wartośći $x_0$, nawet jeśli liczylibyśmy 
średnia z np. 26 okresów (dla $n$=26), co przeczyłoby naszej intuicji.
Moglibyśmy przyjać, że $EMA_{n}(0) = x_0$, lecz dla poprawności obliczeń
w implementacji przyjąłem liczenie dopiero od $i = n+1$, a więc zaczynając od
$EMA_n(n+1) = avg(x_0, x_1, ..., x_n)$ (wartosc dla $i=n+1$ jest równa średniej pierwszych $n$
elementów).
Wskaźnik zostal zaimplementowany w jezyku Python z wykorzystaniem biblioteki
pandas do przechowywania i odczytu danych. 

Jako dane testowe przyjąłem notowania BTC/USD (1.01.2021 - 19.01.2024) oraz
NVDA (29.06.2020 - 3.12.2024). W ten sposób mogę porównać działanie wskaźnika
na szybko zmieniajacych się rynkach kryptowalut oraz na bardziej stabilnych
rynku akcji. Używane dane posiadały więcej danych niż było wskazane, więc 
zostały obcięte do ok. 1100 rekordów.

Dane te pobrane zostały z serwisu kaggle.com w formacie CSV, wykorzystane były 
do obliczenia wskaźnika MACD oraz do wygenerowania wykresów. 

\pagebreak

\section{Analiza notowań i MACD}
\subsection{BTC/USD}

\begin{figure}[H]
  \centering
  \includegraphics[width=0.95\textwidth]{./figures/BTCbuy_sell.png}
  \caption{Wykres notowań BTC/USD z sygnałami buy/sell}
\end{figure}

Wykres notowań BTC (rys. 1) przedstawwia wielkie wahania cen kryptowaluty -
charakterystyczne dla tego rynku.

\begin{figure}[H]
  \centering
  \includegraphics[width=0.95\textwidth]{./figures/BTCmacd.png}
  \caption{Wykres wskaźnika MACD dla BTC/USD}
\end{figure}

Łatwo zauwazyć, że wskaźnik MACD (rys. 2) momentami za szybko reaguje na zmiany
cen, co może prowadzić do nieoptymalnych sygnałów kupna czy sprzedaży (np. około
stycznia 2022). 

\begin{figure}[H]
  \centering
  \includegraphics[width=0.95\textwidth]{./figures/BTCbuy_sell_subset_bad.png}
  \caption{Wykres notowań BTC/USD z sygnałami buy/sell - przybliżenie}
\end{figure}

Tą sytuację lepiej widać na wykresie zbliżonym (rys. 3). Po zakupie w okolicach 
dnia 27.05.2023, wskaźnik nie sugeruje sprzedaży po wzroście ceny z ok. 27 tys.
USD do ok. 28.5 tys. USD, zamiast tego w oczekiwaniu na większy przyrost ceny
napotkał nagły, jednodniowy spadek ceny do ok. 25.75 tys. USD, gdzie sugeruje
sprzedanie. Już nastepnego dnia cena wraca do ok. 27.5 tys. USD, powodując
wykazanie sygnału kupna, co kończy się kolejnym spadkiem ceny nastepnego dnia i
sprzedaniem zakupionych wcześniej aktywów już po jednym dniu.

\begin{figure}[H]
  \centering
  \includegraphics[width=0.95\textwidth]{./figures/BTCmacd_subset_bad.png}
  \caption{Wykres wskaźnika MACD dla BTC/USD - przybliżenie}
\end{figure}

\pagebreak

\begin{figure}[H]
  \centering
  \includegraphics[width=0.95\textwidth]{./figures/BTCbuy_sell_subset_saved.png}
  \caption{Wykres notowań BTC/USD z sygnałami buy/sell - przybliżenie}
\end{figure}


Jednak szybka reakcja wskaźnika na zmiany cen może ratować inwestora w przypadku
spanikowania rynku, tak jak na przykładzie rys. 5, gdzie w okolicach 7.05.2022
wskaźnik szybko zareagował na spadek ceny, co oszczędziło inwestorowi sporej
straty. Podobna sytuacja miala miejsce w okolicach 13.06.2022 (równiez rys. 5).


\begin{figure}[H]
  \centering
  \includegraphics[width=0.95\textwidth]{./figures/BTCmacd_subset_saved.png}
  \caption{Wykres wskaźnika MACD dla BTC/USD - przybliżenie}
\end{figure}

\pagebreak

\begin{figure}[H]
  \centering
  \includegraphics[width=0.95\textwidth]{./figures/BTCbuy_sell_subset_good.png}
  \caption{Wykres notowań BTC/USD z sygnałami buy/sell - przybliżenie}
\end{figure}

Mimo tego, wskaźnik MACD nie jest idealny i nie zawsze daje optymalne sygnały
kupna i sprzedaży, jest w stanie pomóc inwestorowi w zarządzaniu ryzykiem.
Cena aktywów rośnie, co sprawia, że wskaźnik MACD daje sygnały kupna, po czym
lekko spada, co powoduje sygnał sprzedazy, wychodzi to dwukrotnie na korzyść
inwestora (rys. 7).

\begin{figure}[H]
  \centering
  \includegraphics[width=0.95\textwidth]{./figures/BTCmacd_subset_good.png}
  \caption{Wykres wskaźnika MACD dla BTC/USD - przybliżenie}
\end{figure}

\pagebreak

\subsection{NVDA}

\begin{figure}[H]
  \centering
  \includegraphics[width=0.95\textwidth]{./figures/NVDAbuy_sell.png}
  \caption{Wykres notowań NVDA z sygnałami buy/sell}
\end{figure}

Na wykresie notowań NVDA (rys. 9) widać wiekszą stabilność cen akcji w
porównaniu do kryptowaluty. Nietrudno też zauważyć wręcz wykładniczy wzrost
cen akcji w okresie 2023-2024 spowodowany bańką sztucznej inteligencji. 
W takich warunkach wskaźnik MACD może byc bardziej przydatny niż w przypadku
BTC. Przez stabilność ceny akcji wskaźnik MACD daje sygnały kupna i sprzedaży
znacznie cześćiej niż w przypadku BTC.

\begin{figure}[H]
  \centering
  \includegraphics[width=0.95\textwidth]{./figures/NVDAmacd.png}
  \caption{Wykres wskaźnika MACD dla NVDA}
\end{figure}

\pagebreak

Jak widać na rys. 10, wskaźnik MACD dla NVDA daje sygnały kupna i sprzedaży
częściej w stabilnych warunkach rynkowych (linie MACD i SIGNAL w takich
warunkach łatwiej się przecinają). W późniejszych okresach sygnały sa wysyłane 
bardziej optymalnie, gdyż wzrosty i spadki sa wolniejsze i bardziej
przewidywalne. 

\begin{figure}[H]
  \centering
  \includegraphics[width=0.95\textwidth]{./figures/NVDAbuy_sell_subset.png}
  \caption{Wykres notowań NVDA z sygnałami buy/sell - przybliżenie}
\end{figure}

Na przybliżeniu wykresu notowań NVDA (rys. 11) widać, ze wskaźnik MACD
dla stabilnych i wypoziomowanych cen akcji daje sygnały kupna i sprzedaży 
niekoniecznie optymalne, przez łatwe wyprowadzenie go z jego toru i spowodowanie
przecięcia linii MACD i SIGNAL.

\begin{figure}[H]
  \centering
  \includegraphics[width=0.95\textwidth]{./figures/NVDAmacd_subset.png}
  \caption{Wykres wskaźnika MACD dla NVDA - przybliżenie}
\end{figure}

\pagebreak

\section{Symulacja inwestycji}

\subsection{BTC/USD}

Dla symulacji inwestycji przyjalem, ze inwestor zaczyna z 1000 jednostkami
wartosci poczatkowej. Gdy wskaźnik MACD sugeruje kupno, inwestor kupuje
tyle jednostek, ile jest w stanie ze swoim kapitałem. Analogicznie, gdy 
wskaźnik sugeruje sprzedaz, inwestor sprzedaje wszystkie swoje jednostki.

\begin{figure}[H]
  \centering
  \includegraphics[width=0.95\textwidth]{./figures/BTCtransaction_result.png}
  \caption{Symulacja inwestycji w BTC/USD}
\end{figure}

Na rys. 13 widać, ze inwestor zaczynający z 1000 jednostkami kapitału stosujac
algorytm opisany wcześniej, skończy z 1343.17 jednostkami, co daje zysk
\textbf{34.32\%}.

Z 41 wykonanych par transakcji (kupno/sprzedaz), 12 bylo zyskownych, a 29
stratnych. Na rys. 1 widać, że punktami zagęszczenia transakcji
byly okresy względnie stabilnych cen.

\subsubsection{Obliczanie EMA}
We wstępie teorytycznym poruszylem temat obliczania wykładniczej średniej
kroczącej. Wspomniałem o zakładaniu, że $EMA_{n}(0) = x_0$, lecz
zaimplementowałem bliższy prawdzie algorytm, który zaczyna obliczanie
średniej dopiero od $i = n+1$. Oto rezultaty transakcji dla obydwu przypadków:

% EMA(0) = x_0  ---> 1334.87
% EMA(n+1) = avg(x_0, x_1, ..., x_n)  ---> 1343.17

\begin{itemize}
  \item $EMA_{n}(0) = x_0 \rightarrow 1334.87$  
  \item $EMA_{n}(n+1) = avg(x_0, x_1, ..., x_n) \rightarrow 1343.17 $ 
\end{itemize}

Różnica wynosi 1,7 jednostki, zmniejsza sie ona wraz z wielkością danych.
Nastepne podane wyniki sa obliczane poprawną metodą (zaczynając od n+1).

\smallbreak
Dla podanych wcześniej przybliżeń (rys. 3 i 5) otrzymujemy następujące wyniki:

\begin{itemize}
  \item 15.05-20.06(rys. 3)  $1000 \rightarrow 990.88$
  \item 17.04-16.07(rys. 5) $1000 \rightarrow 832.15$ (uratowana
    potencjalna strata 45.8\%)
  \item 27.01-24.03(rys. 7)  $1000 \rightarrow 1468.66$
\end{itemize}



\subsection{NVDA}

\begin{figure}[H]
  \centering
  \includegraphics[width=0.95\textwidth]{./figures/NVDATransaction_result.png}
  \caption{Symulacja inwestycji w NVDA}
\end{figure}

W przypadku NVDA (rys. 14) wykres balansu inwestora przypomina 
wykres cen akcji sam w sobie. Wyraźnie widać tu końcowy przyrost kapitału.
Zaczynając z 1000 jednostkami kapitału, inwestor kończy z 4226.82 jednostkami.
Zysk wynosi \textbf{322.68\%}. Dla fragmentu wykresu (rys. 11) otrzymujemy przy początkowym kapitale 1000
jednostek rezultat w postaci 802.55 jednostek, co daje stratę 19.74\%.

Z 42 wykonanych par transakcji, 10 bylo zyskownych, a 32 stratne.

\subsubsection{Mozliwe optymalizacje algorytmu}

Algorytm sprzedawania wszystkiego gdy tylko pojawia się sygnał kup/sprzedaj 
spowodowany MACD często prowadził do sytuacji, w której kupował wiele razy dzień
po dniu, powodując straty. Do zapobiegnięcia takiej panice możnaby
zaimplementować dodatkowe warunki dot. ostatniej transakcji, np. żeby nie
kupować/sprzedawać w ciagu 3 dni od ostatniego przecięcia.

\section{Wnioski}

Wskaźnik MACD w przypadku BTC/USD nie był tak wydajny jak w przypadku NVDA,
lecz mimo wszystko w obydwu przypadkach przyniósł zyski. W przypadku BTC/USD
zysk wyniosł 34.32\%, a w przypadku NVDA 322.68\%. W obydwu zestawach danych
wskaźnik w wiekszości przypadkow nie trafial optymalnie z sygnałami kupna
czy sprzedazy, jedynie ok. 25\% transakcji bylo zyskownych. Latwo bylo jednak 
odrobic straty dzieki wysokim skokom cen aktywow. Dwa zestawy danych ktore
wybralem do analizy sa znane jako jedne z generujacych najwieksze zyski.
Bitcoin jest znany z dużych skoków cen, a NVDA z dużego i w miarę stabilnego 
wzrostu cen akcji. MACD zdecydowanie lepiej sprawdził się w przypadku NVDA,
gdyż w przypadku BTC za panikował i gdyby nie powolny i stosunkowo stabilny 
wzrost cen od roku 2023, inwestor mógłby stracić sporo kapitału.

Wskaźnik MACD wciąż pozostaje jednym z najpopularniejszych wskaźnikow analizy 
technicznej, lecz nie jest on na tyle dobry by całkowicie oddać mu stery.
Jest on jednak w stanie pomóc inwestorowi w zarządzaniu ryzykiem i sygnalizować
potencjalne zmiany trendu na rynku.


\printbibliography

\end{document}


