\documentclass[a4paper,12pt]{article}  % or "report" for larger documents

\usepackage[utf8]{inputenc}  % UTF-8 encoding
\usepackage{amsmath, amssymb}  % Math support
\usepackage{float}
\usepackage{graphicx}  % Insert images
\usepackage{hyperref}  % Clickable links
\usepackage{biblatex}  % Bibliography management
\addbibresource{references.bib}  % Reference file
\renewcommand{\figurename}{rys.}

\title{Metody numeryczne, projekt1:\\ wskaznik MACD}
\author{Franciszek Fabinski - s197797}
\date{\today}

\begin{document}

\maketitle  % Generates title page


\section{Wstep teoretyczny}
Wskaznik MACD (Moving Average Convergence Divergence) jest jednym z
najpopularniejszych wskaznikow analizy technicznej. Stworzony przez 
Geralda Appela w latach 70-tych XX wieku, mial pomagac inwestorom
w analizie trendow na rynkach finansowych. 

Wskaznik MACD wynika z porownania dwoch wykladniczych srednich kroczacych (EMA), przewaznie
jendej krotszej (szybciej reagujacej na zmiany cen) i jednej dluzszej
(reagujacej wolniej). Pozwala to na wykrycie zmian trendu na rynku.

Poza sama linia MACD, czesto stosuje sie tez linie sygnalowa powstajaca
z wyznaczenia wykladniczej sredniej kroczacej z linii MACD. Przeciecie tych linii 
sugeruje sygnal kupna lub sprzedazy.

\section{Formalizm matematyczny i dane testowe}
Zaimplementowany wskaznik MACD z definicji korzysta z dwoch srednich kroczacych:
\begin{equation}
    MACD = EMA_{12} - EMA_{26}
\end{equation}
gdzie $EMA_{n}$ to srednia kroczaca z $n$ okresow.
W przypadku mojej pracy jeden okres jest rowny jednemu dniu.
Linia SIGNAL jest za to okreslona jako
\begin{equation}
    SIGNAL = EMA_{9}(MACD)
\end{equation}


\pagebreak


Do obliczenia wykladniczej sredniej kroczacej z $n$ okresow korzystam z nizej
podanego wzoru:
\begin{equation}
    EMA_{n}(i) = \alpha \cdot x_i + (1 - \alpha) \cdot EMA_{n}(i-1)
\end{equation}


Dane rownanie jestesmy w stanie przeksztalcic do postaci jawnej:

\begin{equation}
  EMA_n(i) = \frac{x_i + (1-\alpha)x_{i-1} + (1-\alpha)^2 x_{i-2} + ... +
  (1-\alpha)^i x_0}{1 + (1-\alpha) + (1-\alpha)^2 + ... + (1-\alpha)^i}
\end{equation}

gdzie:
\begin{itemize}
  \item $EMA_{n}(i)$ to wartosc sredniej kroczacej z $n$ okresow w $i$-tym okresie
  \item $x_i$ to wartosc z danego okresu
  \item $\alpha = \frac{2}{n+1}$
\end{itemize}


Postac jawna rownania (4) sugeruje, ze jestesmy w stanie obliczyc 
$EMA_{n}(0)$ juz z jednej wartosci $x_0$, nawet jesli liczylibysmy 
srednia z np. 26 okresow (dla $n$=26), co przeczyloby naszej intuicji.
Moglibysmy przyjac, ze $EMA_{n}(0) = x_0$, lecz dla poprawnosci obliczen
w implementacji przyjalem liczenie dopiero od $i = n+1$, a wiec zaczynajac od
$EMA_n(n+1) = avg(x_0, x_1, ..., x_n)$ (wartosc dla $i=n+1$ jest rowna sredniej pierwszych $n$
elementow).
Wskaznik zostal zaimplementowany w jezyku Python z wykorzystaniem biblioteki
pandas do przechowywania i odczytu danych. 

Jako dane testowe przyjalem notowania BTC/USD (1.01.2021 - 19.01.2024) oraz
NVDA (29.06.2020 - 3.12.2024). W ten sposob moge porownac dzialanie wskaznika
na szybko zmieniajacych sie rynkach kryptowalut oraz na bardziej stabilnych
rynku akcji. Uzywane dane posiadaly wiecej danych niz bylo wskazane, wiec 
zostaly obciete do ok. 1100 rekordow.

Dane te pobrane zostaly z serwisu kaggle.com w formacie CSV, wykorzystane byly 
do obliczenia wskaznika MACD oraz do wygenerowania wykresow. 

\pagebreak

\section{Analiza notowan i MACD}
\subsection{BTC/USD}

\begin{figure}[H]
  \centering
  \includegraphics[width=0.95\textwidth]{./figures/BTCbuy_sell.png}
  \caption{Wykres notowan BTC/USD z sygnalami buy/sell}
\end{figure}

Wykres notowan BTC (rys. 1) przedstawwia wielkie wahania cen kryptowaluty -
charakterystyczne dla tego rynku.

\begin{figure}[H]
  \centering
  \includegraphics[width=0.95\textwidth]{./figures/BTCmacd.png}
  \caption{Wykres wskaznika MACD dla BTC/USD}
\end{figure}

Latwo zauwazyc, ze wskaznik MACD (rys. 2) momentami za szybko reaguje na zmiany
cen, co moze prowadzic do nieoptymalnych sygnalow kupna czy sprzedazy (np. okolo
stycznia 2022). 

\begin{figure}[H]
  \centering
  \includegraphics[width=0.95\textwidth]{./figures/BTCbuy_sell_subset_bad.png}
  \caption{Wykres notowan BTC/USD z sygnalami buy/sell - przyblizenie}
\end{figure}

Ta sytuacje lepiej widac na wykresie zblizonym (rys. 3). Po zakupie w okolicach 
dnia 27.05.2023, wskaznik nie sugeruje sprzedazy po wzroscie ceny z ok. 27 tys.
USD do ok. 28.5 tys. USD, zamiast tego w oczekiwaniu na wiekszy przyrost ceny
napotkal nagly, jednodniowy spadek ceny do ok. 25.75 tys. USD, gdzie sugeruje
sprzedanie. Juz nastepnego dnia cena wraca do ok. 27.5 tys. USD, powodujac
wykazanie sygnalu kupna, co konczy sie kolejnym spadkiem ceny nastepnego dnia i
sprzedaniem zakupionych wczesniej aktywow juz po jednym dniu.

\begin{figure}[H]
  \centering
  \includegraphics[width=0.95\textwidth]{./figures/BTCmacd_subset_bad.png}
  \caption{Wykres wskaznika MACD dla BTC/USD - przyblizenie}
\end{figure}

\pagebreak

\begin{figure}[H]
  \centering
  \includegraphics[width=0.95\textwidth]{./figures/BTCbuy_sell_subset_saved.png}
  \caption{Wykres notowan BTC/USD z sygnalami buy/sell - przyblizenie}
\end{figure}


Jednak szybka reakcja wskaznika na zmiany cen moze ratowac inwestora w przypadku
spanikowania rynku, tak jak na przykladzie rys. 5, gdzie w okolicach 7.05.2022
wskaznik szybko zareagowal na spadek ceny, co oszczedzilo inwestorowi sporej
straty. Podobna sytuacja miala miejsce w okolicach 13.06.2022 (rowniez rys. 5).


\begin{figure}[H]
  \centering
  \includegraphics[width=0.95\textwidth]{./figures/BTCmacd_subset_saved.png}
  \caption{Wykres wskaznika MACD dla BTC/USD - przyblizenie}
\end{figure}

\pagebreak

\begin{figure}[H]
  \centering
  \includegraphics[width=0.95\textwidth]{./figures/BTCbuy_sell_subset_good.png}
  \caption{Wykres notowan BTC/USD z sygnalami buy/sell - przyblizenie}
\end{figure}

Mimo tego, wskaznik MACD nie jest idealny i nie zawsze daje optymalne sygnaly
kupna i sprzedazy, jest w stanie pomoc inwestorowi w zarzadzaniu ryzykiem.
Cena aktywow rosnie, co sprawia ze wskaznik MACD daje sygnaly kupna, po czym
lekko spada, co powoduje sygnal sprzedazy, wychodzi to dwukrotnie na korzysc
inwestora (rys. 7).

\begin{figure}[H]
  \centering
  \includegraphics[width=0.95\textwidth]{./figures/BTCmacd_subset_good.png}
  \caption{Wykres wskaznika MACD dla BTC/USD - przyblizenie}
\end{figure}

\pagebreak

\subsection{NVDA}

\begin{figure}[H]
  \centering
  \includegraphics[width=0.95\textwidth]{./figures/NVDAbuy_sell.png}
  \caption{Wykres notowan NVDA z sygnalami buy/sell}
\end{figure}

Na wykresie notowan NVDA (rys. 9) widac wieksza stabilnosc cen akcji w
porownaniu do kryptowaluty. Nietrudno tez zauwazyc wrecz wykladniczy wzrost
cen akcji w okresie 2023-2024 spowodowany banka sztucznej inteligencji. 
W takich warunkach wskaznik MACD moze byc bardziej przydatny niz w przypadku
BTC. Przez stabilnosc ceny akcji wskaznik MACD daje sygnaly kupna i sprzedazy
znacznie czesciej niz w przypadku BTC.

\begin{figure}[H]
  \centering
  \includegraphics[width=0.95\textwidth]{./figures/NVDAmacd.png}
  \caption{Wykres wskaznika MACD dla NVDA}
\end{figure}

\pagebreak

Jak widac na rys. 10, wskaznik MACD dla NVDA daje sygnaly kupna i sprzedazy
czesciej w stabilnych warunkach rynkowych (linie MACD i SIGNAL w takich
warunkach latwiej sie przecinaja). W pozniejszych okresach sygnaly sa wysylane 
bardziej optymalnie, gdyz wzrosty i spadki sa wolniejsze i bardziej
przewidywalne. 

\begin{figure}[H]
  \centering
  \includegraphics[width=0.95\textwidth]{./figures/NVDAbuy_sell_subset.png}
  \caption{Wykres notowan NVDA z sygnalami buy/sell - przyblizenie}
\end{figure}

Na przyblizeniu wykresu notowan NVDA (rys. 11) widac, ze wskaznik MACD
dla stabilnych i wypoziomowanych cen akcji daje sygnaly kupna i sprzedazy 
niekoniecznie optymalne, przez latwe wyprowadzenie go z jego toru i spowodowanie
przeciecia linii MACD i SIGNAL.

\begin{figure}[H]
  \centering
  \includegraphics[width=0.95\textwidth]{./figures/NVDAmacd_subset.png}
  \caption{Wykres wskaznika MACD dla NVDA - przyblizenie}
\end{figure}

\pagebreak

\section{Symulacja inwestycji}

\subsection{BTC/USD}

Dla symulacji inwestycji przyjalem, ze inwestor zaczyna z 1000 jednostkami
wartosci poczatkowej. Gdy wskaznik MACD sugeruje kupno, inwestor kupuje
tyle jednostek, ile jest w stanie ze swoim kapitalem. Analogicznie, gdy 
wskaznik sugeruje sprzedaz, inwestor sprzedaje wszystkie swoje jednostki.

\begin{figure}[H]
  \centering
  \includegraphics[width=0.95\textwidth]{./figures/BTCtransaction_result.png}
  \caption{Symulacja inwestycji w BTC/USD}
\end{figure}

Na rys. 13 widac, ze inwestor zaczynajacy z 1000 jednostkami kapitalu stosujac
algorytm opisany wczesniej, skonczy z 1343.17 jednostkami, co daje zysk
\textbf{34.32\%}.

Z 41 wykonanych par transakcji (kupno/sprzedaz), 12 bylo zyskownych, a 29
stratnych. Na rys. 1 widac, ze punktami zageszczenia transakcji
byly okresy wzglednie stabilne cen.

\subsubsection{Obliczanie EMA}
We wstepie teorytycznym poruszylem temat obliczania wykladniczej sredniej
kroczacej. Wspomnialem o zakladaniu, ze $EMA_{n}(0) = x_0$, lecz
zaimplementowalem blizszy prawdzie algorytm, ktory zaczyna obliczanie
sredniej dopiero od $i = n+1$. Oto rezultaty transakcji dla obydwu przypadkow:

% EMA(0) = x_0  ---> 1334.87
% EMA(n+1) = avg(x_0, x_1, ..., x_n)  ---> 1343.17

\begin{itemize}
  \item $EMA_{n}(0) = x_0 \rightarrow 1334.87$  
  \item $EMA_{n}(n+1) = avg(x_0, x_1, ..., x_n) \rightarrow 1343.17 $ 
\end{itemize}

Roznica wynosi 1,7 jednostki, zmniejsza sie ona wraz z wielkoscia danych.
Nastepne podane wyniki sa obliczane poprawna metoda (zaczynajac od n+1).

\smallbreak
Dla podanych wczesniej przyblizen (rys. 3 i 5) otrzymujemy nastepujace wyniki:

\begin{itemize}
  \item 15.05-20.06(rys. 3)  $1000 \rightarrow 990.88$
  \item 17.04-16.07(rys. 5) $1000 \rightarrow 832.15$ (uratowana
    potencjalna strata 45.8\%)
  \item 27.01-24.03(rys. 7)  $1000 \rightarrow 1468.66$
\end{itemize}



\subsection{NVDA}

\begin{figure}[H]
  \centering
  \includegraphics[width=0.95\textwidth]{./figures/NVDATransaction_result.png}
  \caption{Symulacja inwestycji w NVDA}
\end{figure}

W przypadku NVDA (rys. 14) wykres balansu inwestora przypomina 
wykres cen akcji sam w sobie. Wyraznie widac tu koncowy przyrost kapitalu.
Zaczynajac z 1000 jendostkami kapitalu, inwestor konczy z 4226.82 jednostkami.
Zysk wynosi \textbf{322.68\%}. Dla fragmentu wykresu (rys. 11) otrzymujemy przy poczatkowym kapitale 1000
jednostek rezultat w postaci 802.55 jednostek, co daje strate 19.74\%.

Z 42 wykonanych par transakcji, 10 bylo zyskownych, a 32 stratne.

\subsubsection{Mozliwe optymalizacje algorytmu}

Algorytm sprzedawania wszystkiego gdy tylko pojawia sie sygnal kup/sprzedaj 
spowodowany MACD czesto prowadzil do sytuacji, w ktorej kupowal wiele razy dzien
po dniu, powodujac straty. Do zapobiegniecia takiej paniki moznaby
zaimplementowac dodatkowe warunki dot. ostatniej transakcji, np. zeby nie
kupowac/sprzedawac w ciagu 3 dni od ostatniego przeciecia.

\section{Podsumowanie}

Wskaznik MACD w przypadku BTC/USD nie byl tak wydajny jak w przypadku NVDA,
lecz mimo wszystko w obydwu przypadkach przyniosl zyski. W przypadku BTC/USD
zysk wyniosl 34.32\%, a w przypadku NVDA 322.68\%. W obydwu przypadkach
wskaznik w wiekszosci przypadkow nie trafial optymalnie z sygnalami kupna
czy sprzedazy, jedynie ok. 25\% transakcji bylo zyskownych. Latwo bylo jednak 
odrobic straty dzieki wysokim skokom cen aktywow. Dwa zestawy danych ktore
wybralem do analizy sa znane jako jedne z generujacych najwieksze zyski.
Bitcoin jest znany z duzych skokow cen, a NVDA z duzego i w miare stabilnego 
wzrostu cen akcji. MACD zdecydowanie lepiej sprawdzil sie w przypadku NVDA,
gdyz w przypadku BTC za szybko panikowal i gdyby nie powolny i w miare stabilny 
wzrost cen od roku 2023, inwestor moglby stracic sporo kapitalu.

Wskaznik MACD wciaz pozostaje jednym z najpopularniejszych wskaznikow analizy 
technicznej, lecz nie jest on na tyle dobry by calkowicie oddac mu stery.
Jest on jednak w stanie pomoc inwestorowi w zarzadzaniu ryzykiem i sygnalizowac
potencjalne zmiany trendu na rynku.


\printbibliography

\end{document}


